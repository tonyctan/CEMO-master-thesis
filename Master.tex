\documentclass[a4paper,11pt,UKenglish,twoside,openright]{report}\usepackage[]{graphicx}\usepackage[]{color}
% maxwidth is the original width if it is less than linewidth
% otherwise use linewidth (to make sure the graphics do not exceed the margin)
\makeatletter
\def\maxwidth{ %
  \ifdim\Gin@nat@width>\linewidth
    \linewidth
  \else
    \Gin@nat@width
  \fi
}
\makeatother

\usepackage{Sweavel}



\usepackage{uiothesis}

\title{Identifying School Climate Variables Associated with Students' Financial Literacy Outcomes}
\subtitle{A Cross-Country Comparison\\Using {PISA} 2018 Data}
\author{Tony C. A. Tan}

\begin{document}

% Input faculty etc info
\duoforside[
    fac={Faculty of Educational Sciences},
    dept={Centre for Educational Measurement},
    program={Assessment, Measurement and Evaluation},
    date={Spring 2021},
%  printer={X-press printing house},
    short
]

% Change page numbering to roman
\pagenumbering{roman}

% Create an acknowledgement page
\thispagestyle{empty}

% This is where I generated the Chinese characters
%//ref http://www.akuziti.com/mb/
% I used font 24.Huawen Xingkai and 100 pixel
\vspace*{\fill}
\begin{figure*}[h]
    \includegraphics[width=\textwidth]{./Figures/To-parents.png}
\end{figure*}
\begin{flushright}
To my parents
\end{flushright}
\vspace*{\fill}
%\newpage
\clearpage
\thispagestyle{empty}

\vspace*{3cm}

\begin{quote}
    \calligra\huge      % Changes the font to caligraphy and huge size.
\hyphenchar\font=-1     % Stops LaTeX from splitting/hyphenating words
Study hard what interests you the most in the most undisciplined, irreverent and original manner possible.
\end{quote}

\begin{figure*}[h]
    \flushright
    \includegraphics[width=0.50\textwidth]{./Figures/Feynman-Signature.jpg}
\end{figure*}
\vspace*{-1cm}

%\begin{flushright}
%--- Richard P. Feynman
%\end{flushright}

\setcounter{page}{0}
% End of acknowledgement page

% Put Table of Content here
\tableofcontents

% Put a List of Tables here
\listoftables

% Put a List of Figures here
\listoffigures



% APA7 Rule 2.21 Line Spacing
%\onehalfspacing
\doublespacing

% Acknowledgement

\chapter*{Acknowledgement}
\label{Ac}

Thank-you goes to

% Popular Abstract

\chapter*{Popular Abstract}
\label{Ab.0}

This is a press release style abstract.

% Abstract

\chapter*{Abstract}
\label{Ab.1}

Repeated economic crises in recent memory have exposed the harsh consequences of financial \emph{illiteracy} shared by high proportions of the general population. Policy makers experienced little resistance when identifying youth as the most effective group for bringing about improvement in citizens' ability to engage with economic and financial matters, but opinions quickly diverge over the optimal approaches for achieving such targeted outcome. Existing literature frequently reports the importance of family environment in cultivating students' financial literacy through the process of ``financial socialisation'' -- [definition goes here] (reference). Such practice, however, encounters interrogation by educators over equity concerns should families remain the main arena for financial literacy development. Schools play vital roles in alleviating inequality in accessing education and training in general but scarce research so far has been devoted into identifying the specific classroom factors that are most effective in advancing students' financial literacy outcomes. The current study therefore attempts to contribute to this enquiry by investigating the relationship between school climate variables and students' financial literacy achievement with an aim of stimulating policy debate about the levers and instruments available to education interventionists for the purpose of improving young people's financial literacy and preparedness as they step into an increasingly uncertain world. Using the 2018 PISA dataset, this paper employs a three-level hierarchical model to conduct cross-country comparisons to highlight school climate variables that are most strongly associated with high financial literacy outcomes.

% End of front matter. None of the pages so far counts towards the page limit.
\clearpage
\thispagestyle{empty}

% Restart page number. Page count starts here.
\pagenumbering{arabic}
\setcounter{page}{0} % Do not swap these two command. Need new chapter to be "Open right".

% Chapter 1 Introduction
%<<child="Chapters/Ch1.Rnw", tidy=FALSE, message=FALSE, warning=FALSE>>=
%@

% Chapter 2 Theoretical framework

\chapter{Conceptual Framework}
\label{chp:2}

\section{In-depth definitions of ``financial literacy''}

\subsection{Every term my readers need in order to understand my research question}

\subsection{Survey not only PISA but also alternative definitions, even critiques of such definitions}

\subsection{Any practices that are common in maths/literature but uncommon in financial literacy? Meaning? Implies?}

\newpage

\section{Country-level Financial Knowledge Index}

PISA 2018 financial literacy dataset \parencite{FLdata} provides rich information about students and schools. For the purpose of cross-country comparison, however, the country-level financial literacy information must be addressed separately by the researchers. Earlier attempts such as \textcite{morenoherrero:2018a} approximated this information using a variable ``quality of math and science education'' to control for country-level differences since consensus is yet to emerge about the most appropriate measure for countries' financial knowledge. Inspired by the UN's approach to forming Human Development Indices, a recent publication by \textcite{olivermarquez:2020} proposed a macroeconomic measure for countries' general financial knowledge levels by examining their economic capability, educational training, existing practices in the financial markets as well as incentives to interact with financial products. More specifically, the authors considered a country's economic capability, represented by its GDP per capita, to be a key dimension in bringing about its financial knowledge index (FKI). Secondly, literature converges on the importance of educational training for a country's financial knowledge capability \parencite{oecd:2005}. Thirdly, countries with regular engagement with sophisticated financial products and financial markets should possess higher FKI. Lastly, countries with higher aggregate consumption levels and with ageing populations are likely to possess higher FKI due to more frequent exposure and pressure in retirement provision, respetively. Macroeconomic data needed for these computations can be sourced from the World Bank \parencite{worldbank:data} and the United Nations' \textit{Human Development Reports} \parencite{hdr:data}.

Combining individual and institutional data cources can be a productive approach in international large-scale assessment (ILSA) research. According to the framework for comparative education analyses \parencite{bray:1995}, this project extends education outcome measures to a country level, addresses the aspect of society and labour market, and relates countries' entire populations to ILSA research \parencite{strietholt:2018}. By combining education outcome data with countries' economic performance indicators, this project remains most comparable to \textcite{hanushek:2012a}---while these authors looked into the relationship between countries' education achievement and their GDP growth, the current investigation highlights how countries' GDP, along with other macroeconomic practices, in turn systematically impacts on their youth's educational performance.

\newpage

\ltable{tab:Missing}{Percentages of Missing Values}{
      \begin{tabular}{ccrrrrrrrrrrrrrrrrrr}
      \toprule
      CNT   &       & \multicolumn{1}{c}{\rotatebox{90}{\texttt{MALE}}} & \multicolumn{1}{c}{\rotatebox{90}{\texttt{IMMI1GEN}}} & \multicolumn{1}{c}{\rotatebox{90}{\texttt{IMMI2GEN}}} & \multicolumn{1}{c}{\rotatebox{90}{\texttt{ESCS}}} & \multicolumn{1}{c}{\rotatebox{90}{\texttt{FCFMLRTY}}} & \multicolumn{1}{c}{\rotatebox{90}{\texttt{FLCONFIN}}} & \multicolumn{1}{c}{\rotatebox{90}{\texttt{PERFEED}}} & \multicolumn{1}{c}{\rotatebox{90}{\texttt{TEACHINT}}} & \multicolumn{1}{c}{\rotatebox{90}{\texttt{FLSCHOOL}}} & \multicolumn{1}{c}{\rotatebox{90}{\texttt{DISCRIM}$^\dagger$}} & \multicolumn{1}{c}{\rotatebox{90}{\texttt{BELONG}}} & \multicolumn{1}{c}{\rotatebox{90}{\texttt{BULLY}}} & \multicolumn{1}{c}{\rotatebox{90}{\texttt{FLFAMILY}}} & \multicolumn{1}{c}{\rotatebox{90}{\texttt{CURSUPP}$^\dagger$}} & \multicolumn{1}{c}{\rotatebox{90}{\texttt{PASCHPOL}$^\dagger$}} & \multicolumn{1}{c}{\rotatebox{90}{\texttt{STRATIO}}} & \multicolumn{1}{c}{\rotatebox{90}{\texttt{EDUSHT}}} & \multicolumn{1}{c}{\rotatebox{90}{\texttt{STAFFSHT}}} \\
        \midrule
        BGR   &       & 0     & \cellcolor[rgb]{ 1,  .945,  .945}6 & \cellcolor[rgb]{ 1,  .945,  .945}6 & \cellcolor[rgb]{ 1,  .973,  .973}3 & \cellcolor[rgb]{ 1,  .882,  .882}12 & \cellcolor[rgb]{ 1,  .733,  .733}27 & \cellcolor[rgb]{ 1,  .898,  .898}10 & \cellcolor[rgb]{ 1,  .906,  .906}10 & \cellcolor[rgb]{ 1,  .792,  .792}21 & \cellcolor[rgb]{ 1,  .722,  .722}28 & \cellcolor[rgb]{ 1,  .808,  .808}19 & \cellcolor[rgb]{ 1,  .694,  .694}31 & \cellcolor[rgb]{ 1,  .784,  .784}22 & \cellcolor[rgb]{ 1,  0,  0}100 & \cellcolor[rgb]{ 1,  0,  0}100 & \cellcolor[rgb]{ 1,  .922,  .922}8 & \cellcolor[rgb]{ 1,  .973,  .973}3 & \cellcolor[rgb]{ 1,  .973,  .973}3 \\
        BRA   &       & 0     & \cellcolor[rgb]{ 1,  .949,  .949}5 & \cellcolor[rgb]{ 1,  .949,  .949}5 & \cellcolor[rgb]{ 1,  .98,  .98}2 & \cellcolor[rgb]{ 1,  .878,  .878}12 & \cellcolor[rgb]{ 1,  .667,  .667}34 & \cellcolor[rgb]{ 1,  .91,  .91}9 & \cellcolor[rgb]{ 1,  .925,  .925}8 & \cellcolor[rgb]{ 1,  .796,  .796}21 & \cellcolor[rgb]{ 1,  .643,  .643}36 & \cellcolor[rgb]{ 1,  .776,  .776}23 & \cellcolor[rgb]{ 1,  .6,  .6}40 & \cellcolor[rgb]{ 1,  .765,  .765}24 & \cellcolor[rgb]{ 1,  .831,  .831}17 & \cellcolor[rgb]{ 1,  .816,  .816}19 & \cellcolor[rgb]{ 1,  .882,  .882}12 & \cellcolor[rgb]{ 1,  .941,  .941}6 & \cellcolor[rgb]{ 1,  .933,  .933}7 \\
        CAN$^\dagger$   &       & 0     & \cellcolor[rgb]{ 1,  .933,  .933}7 & \cellcolor[rgb]{ 1,  .933,  .933}7 & \cellcolor[rgb]{ 1,  .953,  .953}5 & \cellcolor[rgb]{ 1,  .894,  .894}11 & \cellcolor[rgb]{ 1,  .851,  .851}15 & \cellcolor[rgb]{ 1,  0,  0}100 & \cellcolor[rgb]{ 1,  0,  0}100 & \cellcolor[rgb]{ 1,  .875,  .875}13 & \cellcolor[rgb]{ 1,  0,  0}100 & \cellcolor[rgb]{ 1,  .925,  .925}8 & \cellcolor[rgb]{ 1,  .859,  .859}14 & \cellcolor[rgb]{ 1,  .867,  .867}14 & \cellcolor[rgb]{ 1,  0,  0}100 & \cellcolor[rgb]{ 1,  0,  0}100 & \cellcolor[rgb]{ 1,  0,  0}100 & \cellcolor[rgb]{ 1,  .988,  .988}2 & \cellcolor[rgb]{ 1,  .984,  .984}2 \\
        CHL   &       & 0     & \cellcolor[rgb]{ 1,  .961,  .961}4 & \cellcolor[rgb]{ 1,  .961,  .961}4 & \cellcolor[rgb]{ 1,  .976,  .976}3 & \cellcolor[rgb]{ 1,  .906,  .906}10 & \cellcolor[rgb]{ 1,  .769,  .769}24 & \cellcolor[rgb]{ 1,  .953,  .953}5 & \cellcolor[rgb]{ 1,  .961,  .961}4 & \cellcolor[rgb]{ 1,  .875,  .875}13 & \cellcolor[rgb]{ 1,  .706,  .706}30 & \cellcolor[rgb]{ 1,  .851,  .851}15 & \cellcolor[rgb]{ 1,  .663,  .663}34 & \cellcolor[rgb]{ 1,  .855,  .855}15 & \cellcolor[rgb]{ 1,  .918,  .918}9 & \cellcolor[rgb]{ 1,  .918,  .918}8 & \cellcolor[rgb]{ 1,  .82,  .82}18 & \cellcolor[rgb]{ 1,  .906,  .906}9 & \cellcolor[rgb]{ 1,  .914,  .914}9 \\
        ESP   &       & 0     & \cellcolor[rgb]{ 1,  .973,  .973}3 & \cellcolor[rgb]{ 1,  .973,  .973}3 & \cellcolor[rgb]{ 1,  .984,  .984}2 & \cellcolor[rgb]{ 1,  .957,  .957}5 & \cellcolor[rgb]{ 1,  .792,  .792}21 & \cellcolor[rgb]{ 1,  .973,  .973}3 & \cellcolor[rgb]{ 1,  .98,  .98}2 & \cellcolor[rgb]{ 1,  .929,  .929}7 & \cellcolor[rgb]{ 1,  .757,  .757}25 & \cellcolor[rgb]{ 1,  .91,  .91}9 & \cellcolor[rgb]{ 1,  .718,  .718}29 & \cellcolor[rgb]{ 1,  .918,  .918}8 & \cellcolor[rgb]{ 1,  0,  0}100 & \cellcolor[rgb]{ 1,  0,  0}100 & \cellcolor[rgb]{ 1,  .89,  .89}11 & \cellcolor[rgb]{ 1,  .949,  .949}5 & \cellcolor[rgb]{ 1,  .945,  .945}6 \\
        EST   &       & 0     & \cellcolor[rgb]{ 1,  .969,  .969}3 & \cellcolor[rgb]{ 1,  .969,  .969}3 & \cellcolor[rgb]{ 1,  .976,  .976}3 & \cellcolor[rgb]{ 1,  .957,  .957}4 & \cellcolor[rgb]{ 1,  .925,  .925}8 & \cellcolor[rgb]{ 1,  .969,  .969}3 & \cellcolor[rgb]{ 1,  .973,  .973}3 & \cellcolor[rgb]{ 1,  .945,  .945}6 & \cellcolor[rgb]{ 1,  .914,  .914}9 & \cellcolor[rgb]{ 1,  .957,  .957}5 & \cellcolor[rgb]{ 1,  .898,  .898}11 & \cellcolor[rgb]{ 1,  .941,  .941}6 & \cellcolor[rgb]{ 1,  0,  0}100 & \cellcolor[rgb]{ 1,  0,  0}100 & 0     & 0     & 0 \\
        FIN   &       & 0     & \cellcolor[rgb]{ 1,  .98,  .98}2 & \cellcolor[rgb]{ 1,  .98,  .98}2 & \cellcolor[rgb]{ 1,  .984,  .984}2 & \cellcolor[rgb]{ 1,  .957,  .957}4 & \cellcolor[rgb]{ 1,  .898,  .898}10 & \cellcolor[rgb]{ 1,  .973,  .973}3 & \cellcolor[rgb]{ 1,  .973,  .973}3 & \cellcolor[rgb]{ 1,  .937,  .937}6 & \cellcolor[rgb]{ 1,  0,  0}100 & \cellcolor[rgb]{ 1,  .937,  .937}6 & \cellcolor[rgb]{ 1,  .894,  .894}11 & \cellcolor[rgb]{ 1,  .933,  .933}7 & \cellcolor[rgb]{ 1,  0,  0}100 & \cellcolor[rgb]{ 1,  0,  0}100 & \cellcolor[rgb]{ 1,  .98,  .98}2 & \cellcolor[rgb]{ 1,  .925,  .925}7 & \cellcolor[rgb]{ 1,  .925,  .925}7 \\
        GEO   &       & 0     & \cellcolor[rgb]{ 1,  .949,  .949}5 & \cellcolor[rgb]{ 1,  .949,  .949}5 & \cellcolor[rgb]{ 1,  .984,  .984}2 & \cellcolor[rgb]{ 1,  .914,  .914}9 & \cellcolor[rgb]{ 1,  .737,  .737}26 & \cellcolor[rgb]{ 1,  .906,  .906}9 & \cellcolor[rgb]{ 1,  .914,  .914}9 & \cellcolor[rgb]{ 1,  .827,  .827}17 & \cellcolor[rgb]{ 1,  0,  0}100 & \cellcolor[rgb]{ 1,  .855,  .855}15 & \cellcolor[rgb]{ 1,  .784,  .784}22 & \cellcolor[rgb]{ 1,  .796,  .796}21 & \cellcolor[rgb]{ 1,  .965,  .965}4 & \cellcolor[rgb]{ 1,  .953,  .953}5 & \cellcolor[rgb]{ 1,  .988,  .988}1 & \cellcolor[rgb]{ 1,  .984,  .984}2 & \cellcolor[rgb]{ 1,  .98,  .98}2 \\
        IDN   &       & 0     & \cellcolor[rgb]{ 1,  .976,  .976}3 & \cellcolor[rgb]{ 1,  .976,  .976}3 & \cellcolor[rgb]{ 1,  .996,  .996}1 & \cellcolor[rgb]{ 1,  .969,  .969}3 & \cellcolor[rgb]{ 1,  .945,  .945}6 & \cellcolor[rgb]{ 1,  .973,  .973}3 & \cellcolor[rgb]{ 1,  .98,  .98}2 & \cellcolor[rgb]{ 1,  .953,  .953}5 & \cellcolor[rgb]{ 1,  .969,  .969}3 & \cellcolor[rgb]{ 1,  .976,  .976}2 & \cellcolor[rgb]{ 1,  .949,  .949}5 & \cellcolor[rgb]{ 1,  .949,  .949}5 & \cellcolor[rgb]{ 1,  0,  0}100 & \cellcolor[rgb]{ 1,  0,  0}100 & \cellcolor[rgb]{ 1,  .773,  .773}23 & \cellcolor[rgb]{ 1,  .863,  .863}14 & \cellcolor[rgb]{ 1,  .863,  .863}14 \\
        ITA   &       & 0     & \cellcolor[rgb]{ 1,  .965,  .965}4 & \cellcolor[rgb]{ 1,  .965,  .965}4 & \cellcolor[rgb]{ 1,  .976,  .976}3 & \cellcolor[rgb]{ 1,  .933,  .933}7 & \cellcolor[rgb]{ 1,  .827,  .827}17 & \cellcolor[rgb]{ 1,  .957,  .957}4 & \cellcolor[rgb]{ 1,  .961,  .961}4 & \cellcolor[rgb]{ 1,  .902,  .902}10 & \cellcolor[rgb]{ 1,  .769,  .769}23 & \cellcolor[rgb]{ 1,  .902,  .902}10 & \cellcolor[rgb]{ 1,  .733,  .733}27 & \cellcolor[rgb]{ 1,  .882,  .882}12 & \cellcolor[rgb]{ 1,  .847,  .847}16 & \cellcolor[rgb]{ 1,  .835,  .835}17 & \cellcolor[rgb]{ 1,  .906,  .906}9 & \cellcolor[rgb]{ 1,  .969,  .969}3 & \cellcolor[rgb]{ 1,  .973,  .973}3 \\
        LTU   &       & 0     & \cellcolor[rgb]{ 1,  .976,  .976}3 & \cellcolor[rgb]{ 1,  .976,  .976}3 & \cellcolor[rgb]{ 1,  .973,  .973}3 & \cellcolor[rgb]{ 1,  .961,  .961}4 & \cellcolor[rgb]{ 1,  .882,  .882}12 & \cellcolor[rgb]{ 1,  .973,  .973}3 & \cellcolor[rgb]{ 1,  .976,  .976}3 & \cellcolor[rgb]{ 1,  .949,  .949}5 & \cellcolor[rgb]{ 1,  .831,  .831}17 & \cellcolor[rgb]{ 1,  .925,  .925}8 & \cellcolor[rgb]{ 1,  .8,  .8}20 & \cellcolor[rgb]{ 1,  .937,  .937}7 & \cellcolor[rgb]{ 1,  0,  0}100 & \cellcolor[rgb]{ 1,  0,  0}100 & 0     & 0     & 0 \\
        LVA   &       & 0     & \cellcolor[rgb]{ 1,  .98,  .98}2 & \cellcolor[rgb]{ 1,  .98,  .98}2 & \cellcolor[rgb]{ 1,  .98,  .98}2 & \cellcolor[rgb]{ 1,  .949,  .949}5 & \cellcolor[rgb]{ 1,  .91,  .91}9 & \cellcolor[rgb]{ 1,  .973,  .973}3 & \cellcolor[rgb]{ 1,  .976,  .976}3 & \cellcolor[rgb]{ 1,  .937,  .937}6 & \cellcolor[rgb]{ 1,  .867,  .867}14 & \cellcolor[rgb]{ 1,  .937,  .937}6 & \cellcolor[rgb]{ 1,  .851,  .851}15 & \cellcolor[rgb]{ 1,  .933,  .933}7 & \cellcolor[rgb]{ 1,  0,  0}100 & \cellcolor[rgb]{ 1,  0,  0}100 & \cellcolor[rgb]{ 1,  .937,  .937}6 & \cellcolor[rgb]{ 1,  .969,  .969}3 & \cellcolor[rgb]{ 1,  .965,  .965}4 \\
        NLD   &       & 0     & \cellcolor[rgb]{ 1,  .973,  .973}3 & \cellcolor[rgb]{ 1,  .973,  .973}3 & \cellcolor[rgb]{ 1,  .98,  .98}2 & \cellcolor[rgb]{ 1,  .973,  .973}3 & \cellcolor[rgb]{ 1,  .953,  .953}5 & \cellcolor[rgb]{ 1,  .976,  .976}3 & \cellcolor[rgb]{ 1,  .976,  .976}2 & \cellcolor[rgb]{ 1,  .965,  .965}4 & \cellcolor[rgb]{ 1,  0,  0}100 & \cellcolor[rgb]{ 1,  .957,  .957}4 & \cellcolor[rgb]{ 1,  .922,  .922}8 & \cellcolor[rgb]{ 1,  .965,  .965}4 & \cellcolor[rgb]{ 1,  0,  0}100 & \cellcolor[rgb]{ 1,  0,  0}100 & \cellcolor[rgb]{ 1,  .894,  .894}11 & \cellcolor[rgb]{ 1,  .957,  .957}5 & \cellcolor[rgb]{ 1,  .957,  .957}5 \\
        PER   &       & 0     & \cellcolor[rgb]{ 1,  .976,  .976}2 & \cellcolor[rgb]{ 1,  .976,  .976}2 & \cellcolor[rgb]{ 1,  .996,  .996}1 & \cellcolor[rgb]{ 1,  .976,  .976}2 & \cellcolor[rgb]{ 1,  .89,  .89}11 & \cellcolor[rgb]{ 1,  .957,  .957}5 & \cellcolor[rgb]{ 1,  .961,  .961}4 & \cellcolor[rgb]{ 1,  .965,  .965}4 & \cellcolor[rgb]{ 1,  .439,  .439}56 & \cellcolor[rgb]{ 1,  .694,  .694}31 & \cellcolor[rgb]{ 1,  .353,  .353}65 & \cellcolor[rgb]{ 1,  .949,  .949}5 & \cellcolor[rgb]{ 1,  0,  0}100 & \cellcolor[rgb]{ 1,  0,  0}100 & \cellcolor[rgb]{ 1,  .976,  .976}2 & \cellcolor[rgb]{ 1,  .996,  .996}0 & \cellcolor[rgb]{ 1,  .996,  .996}0 \\
        POL   &       & 0     & \cellcolor[rgb]{ 1,  .988,  .988}1 & \cellcolor[rgb]{ 1,  .988,  .988}1 & \cellcolor[rgb]{ 1,  .988,  .988}1 & \cellcolor[rgb]{ 1,  .976,  .976}3 & \cellcolor[rgb]{ 1,  .929,  .929}7 & \cellcolor[rgb]{ 1,  .984,  .984}2 & \cellcolor[rgb]{ 1,  .988,  .988}1 & \cellcolor[rgb]{ 1,  .953,  .953}5 & \cellcolor[rgb]{ 1,  .914,  .914}9 & \cellcolor[rgb]{ 1,  .973,  .973}3 & \cellcolor[rgb]{ 1,  .886,  .886}11 & \cellcolor[rgb]{ 1,  .949,  .949}5 & \cellcolor[rgb]{ 1,  0,  0}100 & \cellcolor[rgb]{ 1,  0,  0}100 & 0     & 0     & 0 \\
        PRT   &       & 0     & \cellcolor[rgb]{ 1,  .945,  .945}6 & \cellcolor[rgb]{ 1,  .945,  .945}6 & \cellcolor[rgb]{ 1,  .953,  .953}5 & \cellcolor[rgb]{ 1,  .922,  .922}8 & \cellcolor[rgb]{ 1,  .89,  .89}11 & \cellcolor[rgb]{ 1,  .945,  .945}6 & \cellcolor[rgb]{ 1,  .945,  .945}6 & \cellcolor[rgb]{ 1,  .906,  .906}10 & \cellcolor[rgb]{ 1,  .855,  .855}15 & \cellcolor[rgb]{ 1,  .925,  .925}8 & \cellcolor[rgb]{ 1,  .827,  .827}17 & \cellcolor[rgb]{ 1,  .902,  .902}10 & \cellcolor[rgb]{ 1,  .906,  .906}10 & \cellcolor[rgb]{ 1,  .902,  .902}10 & \cellcolor[rgb]{ 1,  .886,  .886}11 & \cellcolor[rgb]{ 1,  .992,  .992}1 & \cellcolor[rgb]{ 1,  .992,  .992}1 \\
        RUS   &       & 0     & \cellcolor[rgb]{ 1,  .976,  .976}3 & \cellcolor[rgb]{ 1,  .976,  .976}3 & \cellcolor[rgb]{ 1,  .984,  .984}2 & \cellcolor[rgb]{ 1,  .925,  .925}8 & \cellcolor[rgb]{ 1,  .875,  .875}13 & \cellcolor[rgb]{ 1,  .957,  .957}5 & \cellcolor[rgb]{ 1,  .965,  .965}4 & \cellcolor[rgb]{ 1,  .894,  .894}11 & \cellcolor[rgb]{ 1,  .878,  .878}13 & \cellcolor[rgb]{ 1,  .922,  .922}8 & \cellcolor[rgb]{ 1,  .855,  .855}15 & \cellcolor[rgb]{ 1,  .89,  .89}11 & \cellcolor[rgb]{ 1,  0,  0}100 & \cellcolor[rgb]{ 1,  0,  0}100 & \cellcolor[rgb]{ 1,  .973,  .973}3 & \cellcolor[rgb]{ 1,  .976,  .976}3 & \cellcolor[rgb]{ 1,  .976,  .976}3 \\
        SRB   &       & 0     & \cellcolor[rgb]{ 1,  .973,  .973}3 & \cellcolor[rgb]{ 1,  .973,  .973}3 & \cellcolor[rgb]{ 1,  .988,  .988}1 & \cellcolor[rgb]{ 1,  .902,  .902}10 & \cellcolor[rgb]{ 1,  .749,  .749}25 & \cellcolor[rgb]{ 1,  .918,  .918}8 & \cellcolor[rgb]{ 1,  .933,  .933}7 & \cellcolor[rgb]{ 1,  .824,  .824}18 & \cellcolor[rgb]{ 1,  .753,  .753}25 & \cellcolor[rgb]{ 1,  .847,  .847}15 & \cellcolor[rgb]{ 1,  .729,  .729}27 & \cellcolor[rgb]{ 1,  .812,  .812}19 & \cellcolor[rgb]{ 1,  0,  0}100 & \cellcolor[rgb]{ 1,  0,  0}100 & \cellcolor[rgb]{ 1,  .922,  .922}8 & \cellcolor[rgb]{ 1,  .996,  .996}1 & \cellcolor[rgb]{ 1,  .996,  .996}1 \\
        SVK   &       & 0     & \cellcolor[rgb]{ 1,  .98,  .98}2 & \cellcolor[rgb]{ 1,  .98,  .98}2 & \cellcolor[rgb]{ 1,  .988,  .988}1 & \cellcolor[rgb]{ 1,  .961,  .961}4 & \cellcolor[rgb]{ 1,  .886,  .886}12 & \cellcolor[rgb]{ 1,  .965,  .965}4 & \cellcolor[rgb]{ 1,  .976,  .976}3 & \cellcolor[rgb]{ 1,  .929,  .929}7 & \cellcolor[rgb]{ 1,  .859,  .859}14 & \cellcolor[rgb]{ 1,  .945,  .945}6 & \cellcolor[rgb]{ 1,  .831,  .831}17 & \cellcolor[rgb]{ 1,  .922,  .922}8 & \cellcolor[rgb]{ 1,  0,  0}100 & \cellcolor[rgb]{ 1,  0,  0}100 & \cellcolor[rgb]{ 1,  .937,  .937}6 & \cellcolor[rgb]{ 1,  .945,  .945}6 & \cellcolor[rgb]{ 1,  .929,  .929}7 \\
        USA   &       & 0     & \cellcolor[rgb]{ 1,  .976,  .976}3 & \cellcolor[rgb]{ 1,  .976,  .976}3 & \cellcolor[rgb]{ 1,  .984,  .984}2 & \cellcolor[rgb]{ 1,  .973,  .973}3 & \cellcolor[rgb]{ 1,  .945,  .945}6 & \cellcolor[rgb]{ 1,  .984,  .984}2 & \cellcolor[rgb]{ 1,  .988,  .988}1 & \cellcolor[rgb]{ 1,  .965,  .965}4 & \cellcolor[rgb]{ 1,  0,  0}100 & \cellcolor[rgb]{ 1,  .965,  .965}4 & \cellcolor[rgb]{ 1,  .945,  .945}6 & \cellcolor[rgb]{ 1,  .957,  .957}4 & \cellcolor[rgb]{ 1,  0,  0}100 & \cellcolor[rgb]{ 1,  0,  0}100 & \cellcolor[rgb]{ 1,  .847,  .847}16 & \cellcolor[rgb]{ 1,  .906,  .906}10 & \cellcolor[rgb]{ 1,  .906,  .906}10 \\
        \bottomrule
        \end{tabular}
}{Using shades of red in addition to numbers (measured in \%), this table visualises the missing percentages by variable and by country. Variables \texttt{DISCRIM}, \texttt{CURSUPP} and \texttt{PASCHPOL} are no longer pursued in the model because too many countries chose not to respond to these questions. Canada (CAN) is not included due to 100 percent missings on multiple variables. $^\dagger$ marks the country and variables that are excluded from subsequent analyses.}{3}

\newpage

\ptikz{fig:mod}{Three-level Structural Equation Model Predicting Youth's Financial Literacy Outcomes}{
\begin{tikzpicture}[
    latvar/.style={ellipse,draw=black,minimum width=3.5cm,minimum height=1cm},
    manvar/.style={rectangle,draw=black,minimum width=2.5cm},
    mean/.style={fill=black!10!white,regular polygon,regular polygon sides=3},
    ->,>=stealth',semithick,
    bend angle=-45,
    decoration={
        zigzag,
        amplitude=1pt,
        segment length=1mm,
        post=lineto,
        post length=4pt
    }
]

%%%%%%%%%%%%%%%%%%%%%%%%%%%%%%%%%%%%%%%%%%%%%%%%%%%%%%%
%%%%% Level 1
%%%%%%%%%%%%%%%%%%%%%%%%%%%%%%%%%%%%%%%%%%%%%%%%%%%%%%%

% Draw a dashed line to mark Level 1
    \draw[black,dashed,-,line width=2pt] (1.5,5)--(14.85,5);
% Insert level labels
    \node at (0,5) {\textbf{$\m{L1}$: Student}};

% Set input vars (school climate)
    \node[manvar] (A1) at (0,4) {$\texttt{FLSCHOOL}_W$};
    \node[manvar] (S1) at (0,2) {$\texttt{NOBULLY}_W$};
    \node[manvar] (C1) at (0,0) {$\texttt{FLFAMILY}_W$};

% Set mediators (affective vars)
    \node[manvar] (fami1) at (5,4) {$\texttt{FCFMLRTY}$};
    \node[manvar] (conf1) at (5,0) {$\texttt{FLCONFIN}$};

% Set outcome var (financial literacy)
    \node[manvar] (flit1) at (9.5,2) {$\texttt{FLIT}_W$};

% Link input vars with mediators
    \draw[blue,->] (A1.east) to node[above,sloped,pos=0.57] {\press{0.280}{***}{0.017}} (fami1.west);
    \draw[blue,->] (A1.east) to node[above,sloped,pos=0.72] {\press{0.160}{***}{0.008}} (conf1.west);

    \draw[blue,->] (S1.east) to node[above,sloped,pos=0.33] {\press{0.056}{***}{0.005}} (fami1.west);
    \draw[blue,->,dashed] (S1.east) to node[below,sloped,pos=0.25] {\press{0.009}{}{0.010}} (conf1.west);

    \draw[blue,->] (C1.east) to node[below,sloped,pos=0.72] {\press{0.127}{***}{0.012}} (fami1.west);
    \draw[blue,->] (C1.east) to node[below,sloped,pos=0.57] {\press{0.159}{***}{0.008}} (conf1.west);

% Link mediators with outcome var
    \draw[blue,->] (fami1.east) to node[above,sloped,pos=0.55] {\press{0.195}{***}{0.013}} (flit1.west);
    \draw[blue,->] (conf1.east) to node[below,sloped,pos=0.55] {\press{0.082}{***}{0.010}} (flit1.west);

% Link input vars with outcome var
    \draw[red,->] (A1.east) to node[above,sloped,pos=0.45] {\press{-0.074}{***}{0.015}} (flit1.west);
    \draw[red,->] (S1.east) to node[above,sloped,pos=0.45] {\press{0.068}{***}{0.009}} (flit1.west);
    \draw[red,->,dashed] (C1.east) to node[above,sloped,pos=0.45] {\press{-0.003}{}{0.013}} (flit1.west);

% Set control vars
    \node[manvar] (immi1) at (13.5,4) {$\texttt{IMMI1GEN}$};
    \node[manvar] (immi2) at (13.5,3) {$\texttt{IMMI2GEN}$};
    \node[manvar] (male) at (13.5,1) {$\texttt{MALE}$};
    \node[manvar] (escs) at (13.5,0) {$\texttt{ESCS}$};


% Link control vars with outcome var
    \draw[->] (immi1.west) to node[above,sloped,pos=0.45] {\press{-0.039}{**}{0.014}} (flit1.north east);
    \draw[->] (immi2.west) to node[below,sloped,pos=0.4] {\press{-0.021}{*}{0.011}} (flit1.north east);
    \draw[->] (male.west) to node[above,sloped,pos=0.45] {\press{0.023}{*}{0.012}} (flit1.south east);
    \draw[->] (escs.west) to node[below,sloped,pos=0.44] {\press{0.208}{***}{0.019}} (flit1.south east);



% Draw a black dot on output var
    \filldraw[black] (8.25,2) circle (2.5pt);

% Draw residual arrow on output var
    \draw[blue,<-] (6.25,4)--(6.75,4);
    \draw[blue,<-] (6.25,0)--(6.75,0);

    \draw[<-] (10.75,2)--(11.25,2);

% Arc covariances
    \draw[<->] (A1.south) to [bend right] node[above,sloped] {\press{-0.047}{***}{0.010}} (S1.north);
    \draw[<->] (A1.south) to [bend left] node[below,sloped,pos=0.25,rotate=180] {\press{0.228}{***}{0.012}} (C1.north);

    \draw[<->] (S1.south) to [bend right] node[above,sloped] {\press{-0.032}{***}{0.007}} (C1.north);

    \draw[<->] (fami1.south) to [bend right] node[above,sloped] {\press{0.205}{***}{0.013}} (conf1.north);

    \draw[<->] (C1.south) to [bend right=15] node[above] {\press{0.098}{***}{0.017}} (escs.south);

%%%%%%%%%%%%%%%%%%%%%%%%%%%%%%%%%%%%%%%%%%%%%%%%%%%%%%%
%%%%% Level 2
%%%%%%%%%%%%%%%%%%%%%%%%%%%%%%%%%%%%%%%%%%%%%%%%%%%%%%%

% Draw a dashed line to mark Level 2
    \draw[black,dashed,-,line width=2pt] (1.5,13)--(14.85,13);
% Insert level labels
    \node at (0,13) {\textbf{$\m{L2}$: School}};

% Input vars (school climate)
    \node[latvar] (A2) at (3,12) {$\texttt{FLSCHOOL}_B$};
    \node[latvar] (S2) at (3,10) {$\texttt{NOBULLY}_B$};
    \node[latvar] (C2) at (3,8) {$\texttt{FLFAMILY}_B$};
    \node[manvar] (R2) at (3,6) {$\texttt{EDUSHORT}$};

% Control var
    \node[manvar] (st2) at (13.5,6) {$\texttt{STRATIO}$};

% Outcome var (financial literacy)
    \node[latvar] (flit2) at (9.5,9) {$\texttt{FLIT}_{B2}$};

% Link input vars with output var
    \draw[->] (A2.east) to node[above,sloped] {\press{-0.259}{***}{0.071}} (flit2.west);
    \draw[->] (S2.east) to node[above,sloped,pos=0.4] {\press{0.441}{***}{0.065}} (flit2.west);
    \draw[->] (C2.east) to node[below,sloped,pos=0.4] {\press{-0.164}{**}{0.053}} (flit2.west);
    \draw[->] (R2.east) to node[below,sloped] {\press{-0.184}{***}{0.041}} (flit2.west);

% Link control vars with output var
    \draw[->,dashed] (st2.west) to node[above,sloped] {\press{0.007}{}{0.077}} (flit2.south east);

% Draw a black dot on output var
    \filldraw[black] (7.75,9) circle (2.5pt);

% Draw residual arrow on output var
    \draw[<-] (11.25,9)--(11.75,9);

% Arc covariances
    \draw[<->] (A2.south) to [bend right] node[above,sloped] {\press{-0.368}{***}{0.064}} (S2.north);
    \draw[<->] (A2.south) to [bend left] node[below,sloped,pos=0.22,rotate=180] {\press{0.267}{**}{0.084}} (C2.north);
    \draw[<->] (A2.south west) to [bend left] node[above,sloped,rotate=180] {\press{0.058}{*}{0.028}} (R2.north west);

    \draw[<->] (S2.south) to [bend right] node[above,sloped] {\press{0.156}{*}{0.076}} (C2.north);
    \draw[<->,dashed] (S2.south) to [bend left] node[below,sloped,pos=0.75] {\press{-0.049}{}{0.042}} (R2.north);

    \draw[<->,dashed] (C2.south) to [bend right] node[above,sloped] {\press{0.037}{}{0.055}} (R2.north);

%%%%%%%%%%%%%%%%%%%%%%%%%%%%%%%%%%%%%%%%%%%%%%%%%%%%%%%
%%%%% Level 3
%%%%%%%%%%%%%%%%%%%%%%%%%%%%%%%%%%%%%%%%%%%%%%%%%%%%%%%

% Draw a dashed line to mark Level 3
    \draw[black,dashed,-,line width=2pt] (1.5,17)--(14.85,17);
% Insert level labels
    \node at (0,17) {\textbf{$\m{L3}$: Country}};

% Set input vars
    \node[mean] (mean3) at (5,16) {$1$};
    \node[manvar] (fki3) at (5,14) {\texttt{FKI}};

% Set output var
    \node[latvar] (flit3) at (9.5,15) {$\texttt{FLIT}_{B3}$};

% Link input vars with output var
    \draw[->] (mean3.east) to node[above,sloped] {\press{10.731}{***}{1.777}} (flit3.west);
    \draw[->] (fki3.east) to node[below,sloped] {\press{0.699}{***}{0.089}} (flit3.west);

% Draw residual arrow on output var
    \draw[<-] (11.25,15)--(11.75,15);

\end{tikzpicture}
}{This multilevel structural equation model predicts youth's financial literacy outcomes using PISA 2018 data, with mediating effects of familiarity with, and confidence in, financial matters at student-level (\textcolor{red}{direct} and \textcolor{blue}{indirect} pathways). Statistics are standardised regression coefficients. Dashed lines represent nonsignificant relations at $\alpha=0.05$ level. FKI = financial knowledge indices, FLIT = financial literacy. Subscript $_W$ = within, $_B$ = between.\\
$^{*}p<0.05$, $^{**}p<0.01$, $^{***}p<0.001$}
%//ref APA7 Manual Figure 7.7 (p. 239)


% Chapter 3 Methods
%<<child="Chapters/Ch3.Rnw", tidy=FALSE, message=FALSE, warning=FALSE>>=
%@

% Chapter 4 Results
%<<child="Chapters/Ch4.Rnw", tidy=FALSE, message=FALSE, warning=FALSE>>=
%@

% Chapter 5 Discussion
%<<child="Chapters/Ch5.Rnw", tidy=FALSE, message=FALSE, warning=FALSE>>=
%@

% Put Reference here
\newpage
\printbibliography[title=References]

% New CEMO policy: leave tables and figures close to the paragraphs that discussed them.

% Tables
%\newpage
%<<child="Tables/tab.Rnw", tidy=FALSE, message=FALSE, warning=FALSE>>=
%@

% Figures
%\newpage
%<<child="Figures/fig.Rnw", tidy=FALSE, message=FALSE, warning=FALSE>>=
%@

%\newpage

% Insert appendices
\begin{appendices}
    \chapter{\textsc{GDPR} Documentation and Ethical Approval}

%\epigraph{\flushleft --- Do you know a good \textsc{GDPR} consultant?\\ --- Yes.\\ --- Can you pass me their email address?\\ --- No.}

This research project discharges its duty imposed by \textsc{EEA}'s general data protection regulation (\textsc{GDPR}) by following Norwegian Centre for Research Data (\textsc{NSD})'s \href{https://nsd.no/personvernombud/en/notify/notification_test.html}{notification test} on Friday, 11 September 2020. Both \href{https://www.oecd.org/pisa/data/2018database/}{\textsc{PISA} 2018 Database} and the \href{https://data.worldbank.org/}{World Bank Open Data} contain only aggregated and de-personalised datasets with no possibility of back-tracing to any particular participant. Resultantly, no identifiable personal data were collected or used at any stage of this research. The \textsc{NSD}'s assessment letter outlines the agency's decision of not subjecting this project to the \textsc{GDPR} notification. The \textsc{NSD} decision letter also satisfies University of Oslo's \href{https://www.uio.no/english/for-employees/support/research/funding/units/hf/imv/data-ethics/}{ethical approval requirement} and concludes the approval process.

\includepdf[pages=-,fitpaper=true,noautoscale=true]{Appendices/Notification-Test.pdf}

\includepdf[pages=-,fitpaper=true,noautoscale=true]{Appendices/not_subject_to_notification.pdf}


    %\setcounter{secnumdepth}{0}
\chapter{Analysis Code}

\begin{singlespace}

\section{Chapter 1}

There is no analysis code in \cref{chp:1}.

\section{Chapter 2}

\subsection{Data Import} \label{R.import} \footnotesize
\lstinputlisting[language=R,style=vscodeR]{./R/0 Import.R}

\subsection{Missing Pattern Inspection} \label{R.missing} \footnotesize
\lstinputlisting[language=R,style=vscodeR]{./R/1 Missing.R}

\subsection{Financial Knowledge Index} \label{R.fki} \footnotesize
\lstinputlisting[language=R,style=vscodeR]{./R/2 FKI.R}

\subsection{Data Reimport} \label{R.reimport} \footnotesize
\lstinputlisting[language=R,style=vscodeR]{./R/3 Reimport.R}

\end{singlespace}

%    \chapter{Derivation of Country-level Financial Knowledge Indices}

%\epigraph{There are two things you are better off not watching in the making: sausages and\\ econometric estimates.}{Edward Leamer}

\section{Theoretical Foundation}

PISA 2018 financial literacy dataset \parencite{FLdata} provides rich information about students and schools. For the purpose of cross-country comparison, however, the country-level data must be addressed separately by the researchers. \textcite{morenoherrero:2018a}, for instance, introduced a variable ``quality of math and science education'' to control for country-level differences since consensus is yet to emerge about the most appropriate measure for ``countries' financial knowledge''. Inspired by the UN's approach to forming Human Development Indices, a recent publication \textcite{olivermarquez:2020} highlighted four aspects of countries' macroeconomic practices in their attempt to develop country-level financial knowledge indices (FKI).

Oliver-M{\'a}rquez and colleagues consider a country's economic capability, represented by its GDP per capita, to be a key dimension in bringing about its FKI. Secondly, literature converges on the importance of education training for a country's financial knowledge capability \parencite{oecd:2005}. Thirdly, countries with regular engagement with sophisticated financial products and financial markets should possess higher FKI. Lastly, countries with higher aggregate consumption levels and with ageing populations are likely to possess higher FKI due to more frequent exposure and pressure in retirement provision, respetively.

More specifically, \textcite{olivermarquez:2020} suggests using the logarithm of GDP per capital in current international dollars (purchasing power parity adjusted) as a measure for the \texttt{Economic Capability} sub-index. For the \texttt{Education Training} sub-index, the authors consider postgraduate-to-total-tertiary-graduation ratios as a reflection of ``highly skilled'' workforce and the mean years of schooling as a measure of countries' general education levels. For the \texttt{Use} sub-index, gross portfolio equity assets (GPEA) and insurance company assets (ICA) are considered sophisticated financial products countries engage themselves in. Additionally, in order to capture the central role of technology in amplifying the proliferation and use of financial assets, the proportion of Internet users (\textsc{IUS}) enters the definition via
\[ \texttt{Use} = ( \text{GPEA} + \text{ICA} ) ^ \text{IUS}. \]
For the final sub-index \texttt{Need}, the authors define
\[ \texttt{Need} = ( \text{PFA} + \text{AC} ) ^ \text{AGEING}, \]
where \textsc{PFA} is the pension-fund-assets-to-\textsc{GDP} ratio. Aggregate consumption is defined as:
\[ \text{AC} = \frac{2\% \times \text{household final consumption expenditure}}{\text{GDP}}, \]
where the ``$2\%$ rule'' is drawn from \textcite{caliendo:2013} and the proportion of ageing population is computed as
\[ \text{AGEING} = \frac{ \left[ \frac{\text{population}(>65)}{\text{population}(20 \sim 64)} \right]_{2018} - \left[ \frac{\text{population}(>65)}{\text{population}(20 \sim 64)} \right]_{2009} }{ \left[ \frac{\text{population}(>65)}{\text{population}(20 \sim 64)} \right]_{2009} }. \]

\section{Data Collection and Missing Data Treatment}

The data sources for FKI computation are documented in \cref{tab:FKIsource} and its associated notes. The sub-indices \texttt{Educational Training} and \texttt{Use} both contain missing observations for the year 2018. Majority of such missing data appear to be the result of administrative delay, with historic observations available until 2017. It is therefore feasible to conduct time-series forecasts using prior year observations to best approximate 2018 values.

\ltable{tab:FKIsource}{Data Sources for FKI Computation}{
    \begin{tabular}{cclc}
    \toprule
    \multicolumn{1}{c}{Database$^\text{a}$} & Country$^\text{b}$ & \multicolumn{1}{c}{Series} & Time \\
    \midrule
    \rowcolor[rgb]{ .9,  .9,  .9} \multicolumn{4}{c}{Economic Capacity} \\
    WB-dev & 20    & GDP per capita, PPP (current international \$) & 2018 \\
    \rowcolor[rgb]{ .9,  .9,  .9} \multicolumn{4}{c}{Educational Training} \\
    WB-ed & 20 \textbackslash\ Russia & Graduates from ISCED 7 programmes in tertiary education, both sexes (number) & 2013--\textbf{2018} \\
          &       & Graduates from ISCED 8 programmes in tertiary education, both sexes (number) & 2013--\textbf{2018} \\
          &       & Graduates from tertiary education, both sexes (number) & 2013--\textbf{2018} \\
    RS & Russia & PhD (Type 1)$^\text{c}$, PhD (Type 2)$^\text{d}$ & 2018 \\
    RE & Russia & Master (Type 1)$^\text{e}$, Master (Type 2)$^\text{f}$, total tertiary \emph{excluding} PhD$^\text{g}$ & 2018 \\
    HDR & 20    & Dimension = Education; Education = Mean years of schooling (years) & 2018 \\
    \rowcolor[rgb]{ .9,  .9,  .9} \multicolumn{4}{c}{Use} \\
    WB-fin & 20    & Gross portfolio equity assets to GDP (\%) & 2011--\textbf{2018} \\
           &       & Insurance company assets to GDP (\%) & 2011--\textbf{2018} \\
    WB-dev & 20    & Individuals using the Internet (\% of population) & 2009--\textbf{2018} \\
    \rowcolor[rgb]{ .9,  .9,  .9} \multicolumn{4}{c}{Need} \\
    WB-fin & 20 \textbackslash\ Georgia & Pension fund assets to GDP (\%) & 2008--\textbf{2018} \\
    GP & Georgia & Minutes of the meeting of the investment board of the Pension Agency$^\text{h}$ & $\textcolor{white}{^\text{*}}$2019$^\text{*}$ \\
    GS & Georgia & GDP at current prices, billion GEL$^\text{i}$ & 2018 \\
    WB-dev & 20    & Household and NPISHs final consumption expenditure, PPP (current international \$) & 2018 \\
          &       & GDP, PPP (current international \$) & 2018 \\
          &       & Population ages 0--14, male & 2009, 2018 \\
          &       & Population ages 0--14, female & 2009, 2018 \\
          &       & Population ages 15--64, male & 2009, 2018 \\
          &       & Population ages 15--64, female & 2009, 2018 \\
          &       & Population ages 65 and above, male & 2009, 2018 \\
          &       & Population ages 65 and above, female & 2009, 2018 \\
          &       & Population ages 15--19, male (\% of male population) & 2009, 2018 \\
          &       & Population ages 15--19, female (\% of female population) & 2009, 2018 \\
          \bottomrule
    \end{tabular}
}{Sub-indices are shaded in gray. Bold font signifies this year contains missing data.}{3}
\newpage

\begin{singlespace} \small
\begin{itemize}
    \item[$^\text{a}$] WB-dev = \href{https://databank.worldbank.org/source/world-development-indicators}{World Bank -- World development indicators}\\
        WB-ed = \href{https://databank.worldbank.org/source/education-statistics-^-all-indicators}{World Bank -- Education statistics -- All indicators}\\
        WB-fin = \href{https://databank.worldbank.org/source/global-financial-development}{World Bank -- Global financial development}\\
        HDR = \href{http://hdr.undp.org/en/data}{Human Development Reports -- Data}\\
        RS = \href{https://rosstat.gov.ru/}{Russian Federal State Statistic Service}\\
        RE = \href{https://minobrnauki.gov.ru}{Russian Ministry of Education and Science}\\
        GP = \href{https://www.pensions.ge}{Pension Agency of Georgia}\\
        GS = \href{https://www.geostat.ge}{National Statistics Office of Georgia}
    \item[$^\text{b}$] ``20'' = the 20 participating countries in 2018 \textsc{PISA} financial literacy test: Brazil, Bulgaria, Canada, Chile, Estonia, Finland, Georgia, Indonesia, Italy, Latvia, Lithuania, the Netherlands, Peru, Poland, Portugal, Russian Federation, Serbia, Slovak Republic, Spain, and the USA. ``\textbackslash'' = exluding or except
    \item[$^\text{c}$] \href{https://rosstat.gov.ru/storage/mediabank/asp-2(1).xls}{https://rosstat.gov.ru/storage/mediabank/asp-2(1).xls}, Sheet ``\foreignlanguage{russian}{по направлениям подготовки}'', Cell C7 = number of PhD graduates \mbox{(Type 1)}
    \item[$^\text{d}$] \href{https://rosstat.gov.ru/storage/mediabank/asp-3.xls}{https://rosstat.gov.ru/storage/mediabank/asp-3.xls}, Sheet ``\foreignlanguage{russian}{по научным специальностям}'', Cell B7 = number of PhD graduates \mbox{(Type 2)}
    \item[$^\text{e--g}$] \href{https://minobrnauki.gov.ru/common/upload/download/VPO_1_2018.rar}{https://minobrnauki.gov.ru/common/upload/download/VPO{\textunderscore}1{\textunderscore}2018.rar} contains a spreadsheet \textcolor{blue}{\foreignlanguage{russian}{СВОД{\textunderscore}ВПО1{\textunderscore}ВСЕГО}.xls}, Sheet ``P2{\textunderscore}1{\textunderscore}3(1)'', Cell E198 = number of master graduates (Type 1)$^\text{e}$, Cell E410 = number of master graduates (Type 2)$^\text{f}$, Cell E592 = total tertiary graduates \emph{excluding} PhD$^\text{g}$
    \item[$^\text{h}$] \href{https://www.pensions.ge/docs/legislation/investment-board-protocol-4.pdf}{Minutes of the meeting of the investment board of the Pension Agency}, p. 4, no. 3
    \item[$^\text{i}$] \href{https://www.geostat.ge/en/modules/categories/23/gross-domestic-product-gdp}{Gross domestic product (GDP)}, row = GDP at current prices, billion GEL, column = 2018
    \item[$^\text{*}$] Georgia started a \href{https://agenda.ge/en/news/2019/13}{new pension system} on 1 January 2019. Since 2018 was a transitional period with scarce data, 2019 is used as the best approximation for Georgia's pension system for 2018.
\end{itemize}
\end{singlespace}

\clearpage



\subsection{Sub-index \texttt{Educational Training}}

The 2018 archive for the number of master (ISCED 7), PhD (ISCED 8), and total tertiary graduates are incomplete for all participating countries except Georgia, Indonesia and Serbia. \cref{fig:skilled} presents a time series plot of
\[ \texttt{highly skilled} = \frac{\text{number of masters} + \text{number of PhDs}}{\text{total number of tertiary graduates}} \]
and suggests that this ratio is likely to be stable over time, especially between adjacent years. A ``naive forecast'', where the nearest available year's data are to be duplicated for 2018, is applied for \texttt{highly skilled}.

\pfigure{fig:skilled}{Proportion of Postgraduates to Total Tertiary Graduations}{1}{./Figures/skilled.pdf}{``Postgraduate'' is defined as master (ISCED 7) and PhD (ISCED 8) graduates. Countries not shown: GEO, IDN and SRB (2018 data available) and RUS (consult other sources)}{1.75}{1.25}

\subsection{Sub-index \texttt{Use}}

All series involved in calculating this sub-index, GPEA, ICA and IUS, contain missing data. When time series data contain only exponential growth but no underlying trend, a simple exponential smoothing would suffice \parencite{garder:1985}; if trend is present, Holt-Winters method is superior \parencite{chatfield:1978}. \cref{fig:use} facilitates this decision making by plotting both the original and log-transformed versions of GPEA and ICA series. Since curves after log-transformations have slopes, it is prudent to apply the Holt-Winters forecasting method in order to account for possible trends contained in the original series.

\pfigure{fig:use}{Time Series Trend Test}{1}{./Figures/use.pdf}{The time series plots after natural logarithm transformations (bottom panels) are not flat, suggesting the original series (top panels) contain trends. Holt-Winters method therefore is preferred over simple exponential smoothing for 2018 forecasts.}{0.5}{0.85}

The IUS series contains missing data for Canada, Chile and the United States. Similar Holt-Winters procedure is applied to recover 2018 IUS data.

\ltable{tab:FKIraw}{Data Utilised for Computing FKI}{
  \begin{tabular}{cd{3} c d{3}d{1} c d{3}d{3}d{3} c d{3}d{3}d{3}}
    \toprule
    & \multicolumn{1}{c}{Economic Capacity} &       & \multicolumn{2}{c}{Educational Training} &       & \multicolumn{3}{c}{Use} &       & \multicolumn{3}{c}{Need} \\
\cmidrule{2-2}\cmidrule{4-5}\cmidrule{7-9}\cmidrule{11-13}          & \multicolumn{1}{c}{GDP per capita} &       & \multicolumn{1}{c}{Skilled} & \multicolumn{1}{c}{Schooling} &       & \multicolumn{1}{c}{GPEA} & \multicolumn{1}{c}{ICA} & \multicolumn{1}{c}{IUS} &       & \multicolumn{1}{c}{PFA} & \multicolumn{1}{c}{AC} & \multicolumn{1}{c}{AGEING} \\
    \midrule
    BRA   & 9.612 &       & 6.484 & 7.8   &       & 1.683 & 16.259 & 70.434 &       & 11.827 & 1.210  & 0.288 \\
    CHL   & 10.117 &       & 16.371 & 10.4  &       & 51.755 & 25.591 & 89.531 &       & 73.225 & 1.073 & 0.214 \\
    EST   & 10.501 &       & 36.765 & 13    &       & 16.399 & 7.681 & 89.357 &       & 18.012 & 0.876 & 0.163 \\
    GEO   & 9.588 &       & 24.039 & 12.8  &       & 0.784 & 1.469 & 62.718 &       & 0.834 & 1.227 & 0.042 \\
    IDN   & 9.362 &       & 7.771 & 8     &       & 0.636 & 4.612 & 39.905 &       & 1.826 & 1.059 & 0.145 \\
    ITA   & 10.665 &       & 44.771 & 10.2  &       & 57.434 & 51.260 & 74.387 &       & 10.589 & 1.075 & 0.155 \\
    LTU   & 10.487 &       & 28.749 & 13    &       & 9.008 & 5.500   & 79.723 &       & 7.457 & 1.107 & 0.149 \\
    PER   & 9.479 &       & 13.577 & 9.2   &       & 16.027 & 6.505 & 52.540 &       & 22.530 & 1.187 & 0.227 \\
    POL   & 10.368 &       & 36.725 & 12.3  &       & 4.853 & 9.535 & 77.542 &       & 9.838 & 1.085 & 0.355 \\
    PRT   & 10.444 &       & 34.454 & 9.2   &       & 19.353 & 25.579 & 74.661 &       & 8.761 & 1.133 & 0.237 \\
    SVK   & 10.391 &       & 54.417 & 12.6  &       & 10.644 & 8.873 & 80.660 &       & 12.497 & 0.962 & 0.300 \\
    ESP   & 10.609 &       & 33.929 & 9.8   &       & 27.681 & 28.230 & 86.107 &       & 10.235 & 1.044 & 0.186 \\
    USA   & 11.048 &       & 24.825 & 13.4  &       & 55.505 & 30.183 & 84.881 &       & 150.040 & 1.364 & 0.252 \\
    \bottomrule
    \end{tabular}
}{Full variable names: Skilled = Postgraduate to total tertiary ratio; Schooling = Mean year of schooling; GPEA = Gross portfolio to GDP ratio; ICA = Insurance company assets to GDP ratio; IUS = Number of Internet users per 100 population; PFA = Pension fund assets to GDP ratio; AC = 2\% of household final consumption expenditure to GDP ratio; AGEING = Aged-to-productive-population ratio (\% change between 2009 and 2018)}{3}


\subsection{Other Items with Data Concerns}

Russia reported 67.96\% and 61.01\% of its total university degree receipients to be postgraduates for the year 2013 and 2015 respectively (2014 missing). This figure rapidly declines to 41.6\% in 2016 and further down to 25.69\% in 2017. Such volatility goes against the stable patterns shared by most countries in \cref{fig:skilled}, casting doubt on data reliability. Separate investigation is therefore conducted using Russian government archive (Notes c to g in \cref{tab:FKIsource}).

Georgia underwent pension reform in 2018 with fund balance gradually transitioning to State Pension Agency for its official resumption of duty on 1 January 2019. Resultantly, 2018 pension balance for this country is unavailable but to be best appoximated using 2019 official data (Notes h, i and * of \cref{tab:FKIsource}).

\cref{tab:FKIraw} documents the results of the abovementioned data recovery process.

\section{Standardisation, Weights and FKI}

Following \textcite{olivermarquez:2020}'s procedure, all series in \cref{tab:FKIraw} undergo min-max normalisation such that the smallest entry receives a new score of $0.01$ and the biggest number is re-coded to $0.99$. This slight deviation from the original paper (where the min-max normalisation yields $0$ to $1$) is to avoid multiplying a series by zero or raising a base to the power of zero.

Variable weights are calculated following \textcite{olivermarquez:2020}'s recipe to be the inverses of each series' standard deviations. Whereas a sub-index combines more than one series, each weight is further divided by the sum of the constituent weights so that total weights add to one.

FKI is finally computed by taking the geometric mean of all four sub-indices, subject to sub-index-weights similar to variable weights above, as presented in \cref{tab:FKI}.

\ptable{tab:FKI}{FKI and Sub-indices}{
    \begin{tabular}{c c d{3} c d{3}d{3}d{3}d{3}}
    \toprule
        && \multicolumn{1}{c}{FKI}       && \multicolumn{1}{c}{EC}        & \multicolumn{1}{c}{ET}        & \multicolumn{1}{c}{Use}       & \multicolumn{1}{c}{Need}\\
    \midrule
        NLD   &       & 0.957 &       & 0.939 & 0.640 & 1.967 & 1.000 \\
        USA   &       & 0.947 &       & 0.990 & 0.589 & 0.918 & 1.407 \\
        ITA   &       & 0.771 &       & 0.767 & 0.602 & 1.145 & 0.806 \\
        FIN   &       & 0.733 &       & 0.850 & 0.685 & 1.194 & 0.563 \\
        ESP   &       & 0.637 &       & 0.734 & 0.464 & 0.697 & 0.726 \\
        LTU   &       & 0.614 &       & 0.664 & 0.633 & 0.256 & 0.835 \\
        PRT   &       & 0.598 &       & 0.639 & 0.401 & 0.680 & 0.762 \\
        BGR   &       & 0.585 &       & 0.396 & 0.760 & 0.404 & 0.728 \\
        EST   &       & 0.579 &       & 0.672 & 0.747 & 0.282 & 0.574 \\
        SVK   &       & 0.562 &       & 0.608 & 0.924 & 0.322 & 0.440 \\
        POL   &       & 0.559 &       & 0.595 & 0.700 & 0.311 & 0.572 \\
        CHL   &       & 0.552 &       & 0.449 & 0.302 & 0.818 & 0.908 \\
        LVA   &       & 0.547 &       & 0.573 & 0.634 & 0.165 & 0.794 \\
        RUS   &       & 0.449 &       & 0.536 & 0.597 & 0.088 & 0.638 \\
        SRB   &       & 0.424 &       & 0.249 & 0.500 & 0.209 & 0.742 \\
        GEO   &       & 0.419 &       & 0.141 & 0.547 & 0.210 & 0.997 \\
        PER   &       & 0.309 &       & 0.078 & 0.194 & 0.701 & 0.877 \\
        BRA   &       & 0.145 &       & 0.155 & 0.010 & 0.472 & 0.832 \\
        IDN   &       & 0.122 &       & 0.010 & 0.040 & 0.974 & 0.787 \\
    \bottomrule
    \end{tabular}
}{Table sorted in descending order by countries' FKI. FKI = financial knowledge index, EC = Economic Capability, ET = Educational Training.}{2.5}


    \chapter{Multilevel Multiple Imputation}
\label{app:MMI}


%    \include{Appendices/E}
%    \include{Appendices/F}
\end{appendices}

% Put Index here
%\begin{multicols}{2}
\cleardoublepage
\phantomsection
\addcontentsline{toc}{chapter}{Name Index}
\printindex[a]

\cleardoublepage
\phantomsection
\addcontentsline{toc}{chapter}{Subject Index}
\printindex
%\end{multicols}

\end{document}
