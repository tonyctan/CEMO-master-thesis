Financial literacy makes or breaks people’s lives.
In one sense, the world never fully recovered from the 2008 financial crises, which was caused by, sure, greed on the bankers and politicians, but also by collective financial illiteracy shared by a critical mass of mortgage loan consumers. Debt literacy, risk literacy, cost-benefit analyses, all exposed the social wound that too many are incapable of making life-changing financial decisions in well-informed and responsible manners.
Poor financial literacy was observed not only in the USA, but in Germany, in EU countries, in OECD member states, all the way to emerging economies such as India and Indonesia.
Sadly, when science and education failed to win over our youth and people, misinformation would. “Trade wars are good, and easy to win.” twitted then President Trump, and people cheered.
A financially literate population is more resilient to political opportunists. Teaching our young about taxation, tariff, outsourcing, labour market transition and career choices protects not only individuals’ financial security and dignity, but ultimately extends its protective power to the very social fabric that holds us together through informed debate and evidence-based, rational, policy formation at national and international levels.
Foot dragging on financial literacy education is not an option. To the children, to the society, we educators owe a duty of care.
So how much financial capability do our children possess when they approach the end of their compulsory education journey? Have schools been doing enough in cultivating youth’s financial knowledge and confidence? What worked, what improvement can be made? All these questions motivated me into this master thesis.
Since youth shares the overwhelming majority of their wakening hours in school, it is natural to contextualise my investigation in this arena.
[Click for Slide 2]
“What does a good school look and feel like?” Parents are quick in acknowledging one when they see it, but defining school climate has been a fuzzy and fluid work, until Wang & Degol’s (2016) framework came out: in it, the authors catalogued school climate into the following four aspects:
•	Academic: the teaching, learning, classroom, exam element of the school life;
•	Community: how well schools liaison with the social ecology, in particular, with parents;
•	Safety: children are not learning when they do not feel safe; and last,
•	Institutional environment: not only the brick and mortar, tangible, physical endowment a school possesses, but also the personnel resources, overcrowding, high or low student-teacher ratios in the school.
Out of the four domains, the first three are tied closely to individual students while resources are more naturally classified as a school-level variable.
[Click for Slide 3]
Hence, a hierarchical model was proposed for this investigation, where
Financial education in school lessons, FLSCHOOL, was chosen as the indicator for the academic domain of school climate,
Parental involvement in matters of financial literacy, FLFAMILY, also known as “financial socialisation”, was for the community domain,
A safe school should be bully-free, hence, the NOBULLY variable.
All three student-level variables are then aggregated into the school-level, FLSCHOOL-within becomes FLSCHOOL-between, etc., to complete Wang & Degol’s school climate framework.
In addition, I am interested in the process, or, the pathway configuration, of financial literacy formation. Prior literature has suggested the key role affective variables play in crystallising young people’s cognitive capabilities. I therefore included two non-cognitive variables:
•	Familiarity with concepts of finance, FCFMLRTY, and
•	Confidence about financial matters, FLCONFIN,
as mediators to complete the path diagram. Demographic information was drawn in gray, including student’s SES, immigration history and sex as well as school’s student-teacher ratios, all under the advice of prior literature.
To date, PISA is the only large-scale international assessment that explicitly measures students’ financial knowledge, attitude and capabilities. This study used the latest round 2018 data as input, and obtained the following statistical results:
[Click for Slide 4]
Based on a sample size of 107,162 students, nested in 6,631 schools, then further nested in 20 countries, I can be assured that these results are more than chance findings.
My overall impression was that:
•	All four school climate variables were important in explaining variations in 15-year-old’s financial literacy outcomes;
•	Affective variables were shown to have played important mediation role in youth’s financial literacy performance process;
•	Students’ SES carried strong explanatory power along both the direct and indirect pathways;
•	Immigrant students appeared to be underperforming their native peers, and
•	Boys outperformed, on average, but not through better financial knowledge, but through stronger affects especially higher confidence.
All these results were highly consistent with intuition, until I noticed the negative relationship between FLSCHOOL and FLIT, also between FLFAMILY and FLIT.
